\documentclass[11pt]{article}

\usepackage{amsfonts}
\usepackage{amsmath}
\usepackage[total={7in,9in}]{geometry}
\usepackage{amsthm}
\usepackage{graphicx}
\usepackage{lmodern}
\usepackage{hyperref}

\title{Berkeley ZK MOOC homework assignment}
\author{ \\ Morgan Thomas \\ Casper Association \\ morgan@casper.network }


\begin{document}

\maketitle

\section{Question 1}

\subsection{Part (a)}

\textbf{Question.} Give an expression for the multilinear extension
$\widetilde{\mathsf{eq}}_\ell : \{0,1\}^\ell \times \{0,1\}^\ell \to \{0,1\}$, that is defined
as follows:

\begin{equation}
  \mathsf{eq}_\ell(a, b) = \begin{cases}
	  1 & \text{if}\ a = b \\
	  0 & \text{otherwise}
  \end{cases}
\end{equation}

Argue that this polynomial can be evaluated at any point in time $O(\ell)$.

\textbf{Answer.} Let:

\begin{equation}
	\widetilde{\mathsf{eq}}_\ell(a,b) = \prod_{i=1}^\ell 1 - a_i + b_i.
\end{equation}

This polynomial agrees with $\mathsf{eq}$ on all
$(a,b) \in \{0,1\}^\ell \times \{0,1\}^\ell$. It is also multilinear,
as we can see by induction on $\ell$. When $\ell = 1$, it is clearly
multilinear. When $\ell > 1$, we can observe that

\begin{equation}
	\widetilde{\mathsf{eq}}_\ell(a,b) = (1 - a_\ell + b_\ell) \cdot \widetilde{\mathsf{eq}}_{\ell-1} = \widetilde{\mathsf{eq}}_{\ell-1}(a,b) - a_\ell \cdot \widetilde{\mathsf{eq}}_{\ell-1}(a,b) + b_\ell \cdot \widetilde{\mathsf{eq}}_{\ell-1}(a,b),
\end{equation}

and $\widetilde{\mathsf{eq}}_{\ell-1}$ does not refer to $a_\ell$ or $b_\ell$, so
therefore $\widetilde{\mathsf{eq}}_\ell$ is multilinear if $\widetilde{\mathsf{eq}}_{\ell-1}$ is.

These polynomials can be evaluated using $\ell$ additive inverse operations (one on each $a_i$), $2\ell$ addition operations, and $\ell-1$ multiplication operations, so, they can be evaluated using $O(\ell)$ operations. Said operations are $O(1)$ time, so overall these polynomials can be evaluated in $O(\ell)$ time.

\subsection{Part (b)}

\textbf{Question.} Assume that $n$ is a power of two. Give an expression for $\widetilde{\mathsf{mult}}$ that can be evaluated at any point in time $O(\log n)$. The multiplication layer consists of $n = 2^{d-1}$ multiplication gates, where the $j$-th multiplication gate at layer $d - 1$ has both in-neighbors equal to the $j$-th input gate at layer $d$.

\textbf{Answer.} For all positive integers $\ell$ and bit vectors $a,b,c$ of length $\ell$, let:

\begin{equation}
	\mathsf{eq}^3_\ell(a, b, c) =
	\begin{cases}
		1 & \text{if}\ a = b = c, \\
		0 & \text{otherwise}.
	\end{cases}
\end{equation}

Define the multilinear extension $\widetilde{\mathsf{eq}}^3_\ell$ of $\mathsf{eq}^3_\ell$ as follows:

\begin{equation}
	\widetilde{\mathsf{eq}}^3_\ell(a, b, c) =
	\prod_{i=1}^\ell (a_i \cdot b_i \cdot c_i) + ((1 - a_i) \cdot (1 - b_i) \cdot (1 - c_i)).
\end{equation}

It should be easy to check that this is an extension of $\mathsf{eq}^3_\ell$.
It is also multilinear because each term of the big product is multilinear and
no two terms refer to the same variable (and this reasoning can be further done out
as a proof by induction as in the previous answer). Also observe that this expression
can be evaluated in $O(\ell)$ time.

Let $n$ be the number of multiplication gates. Let $d$ be the number of layers
of the circuit. Consider that the number of nodes in the circuit is $2^d-1+n$,
where $2^d-1$ is the number of nodes in a perfect binary tree of height $d-1$,
and $n$ is the number of input nodes. Consdering that $n = 2^{d-1}$, it requires
$d+1$ bits to represent an index into the list of nodes.

Let the nodes be indexed from left to right, starting from the bottom layer. 
Let the bits in the index bit strings be indexed in LSB first notation.
Let $k$ be the number of bits
in $n$. Note that $k = O(\log n)$ and $d+1-n = O(d) = O(\log n)$.

Let:

\begin{equation}
	\begin{array}{rcl}
		\widetilde{\mathsf{mult}}(a, b, c) &=& 
		\widetilde{\mathsf{eq}}^3_k([a_1, ..., a_n], [b_1, ..., b_n], [c_1, ..., c_n]) \\
		&\cdot& \widetilde{\mathsf{eq}}^3_{d+1-n}([b_{n+1}, ..., b_{d+1}], [c_{n+1}, ..., c_{d+1}], [0, ..., 0]) \\
		&\cdot& \widetilde{\mathsf{eq}}_{d+1-n}([a_{n+1}, ..., a_{d+1}], [1, 0, ..., 0]).
	\end{array}
\end{equation}

Observe that this defines the multilinear extension of $\mathsf{mult}(a, b, c)$,
and it can be evaluated in $O(\log n)$ since each of the subscripts on the
equality predicates is $O(\log n)$.

\subsection{Part (c)}

\paragraph{Question.} Assume that $n$ is a power of 2. Give an expression for $\widetilde{\mathsf{add}}$
that can be evaluated at any point in time $O(\log n)$. The addition layer $i$ consists of $2^i$
addition gates, where for $j \in \{0, 1, ..., 2^i - 1\}$, the $j$-th addition gate at layer $i$
has as its in-neighbors gates $2j$ and $2j+1$ at layer $i+1$.

\paragraph{Answer.} Let the nodes be indexed from left to right, starting from the bottom layer
and moving up. Let the indices be represented as bit strings in LSB first notation. Let $k$ be the
number of bits in $n$. Let $d$ be the number of layers in the circuit. As in part (b), the number of
nodes in the circuit is $2^d-1+n$, including $n$ input nodes, where $2^d-1$, the number of nodes in
a perfect binary tree of height $d-1$, is the number of non-input nodes. $k+2$ is the number of bits
in an index into the nodes.

Let $\mathsf{idx}(i, j)$ denote the index of the $j$-th (zero indexed from the left) node in the
$i$-th (zero indexed from the bottom) layer. Concretely,

\begin{equation}
	\begin{array}{rcl}
		\mathsf{idx}(i, j) &=& (2^d-1) - (2^{d-(i-1)}-1) + n + j \\
		&=& 2^d - 2^{d-i+1} + n + j.
	\end{array}
\end{equation}
This formula works because $2^d-1$ is the number of nodes in a perfect binary tree of height $d-1$,
and $2^{d-(i-1)}-1$ is the number of nodes in a perfect binary tree of height $d-1-(i-1)$, which is
in other words the number of nodes above layer $i$ in the circuit. $n$ is the number of nodes in
the bottom layer, and $j$ is the number of nodes in layer $i$ before the node with index
$\mathsf{idx}(i, j)$.

A sum of multilinear polynomials is a multilinear polynomial. Therefore, to compute the multilinear
extension of the wiring predicate $\mathsf{add}$, it suffices to compute the multilinear extension
of the addition wiring predicate for each layer, that is, for all $2 \leq i < d$, the multilinear
extension of

\begin{equation}
	\mathsf{add}_i(a, b, c) =
	\begin{cases}
		1 & \exists j \in [0,2^{d-i}), a = \mathsf{idx}(i, j) \wedge b = \mathsf{idx}(i-1, 2j) \wedge c = \mathsf{idx}(i-1, 2j+1), \\
		0 & \text{otherwise}.
	\end{cases}
\end{equation}

Let these layer-specific wiring predicates be broken down as a product of two predicates, one
stating that an index belongs to a particular layer, and one stating that $b_j = 2a_j$ and $c_j = 2a_j + 1$,
where $a_j$ is the $j$ of $a$, and so forth:

\begin{equation}
	\mathsf{layer}_i(a) =
	\begin{cases}
		1 & \exists j \in [0,2^{d-i}), a = \mathsf{idx}(i, j), \\
		0 & \text{otherwise}.
	\end{cases}
\end{equation}

\begin{equation}
	\mathsf{mask}_i(a) = a \text{\&} (2^{d-i-1}-1).
\end{equation}

Here \& denotes the bitwise AND operation. This bit mask is supposed to extract $j$;
that is, $\mathsf{mask}_i(a) = a_j$ if $a$ is in layer $i$.

\begin{equation}
	\mathsf{lidx}_i(a, b, c) =
	\begin{cases}
		1 & \mathsf{mask}_{i-1}(b) = 2 \cdot \mathsf{mask}_i(a) \wedge \mathsf{mask}_{i-1}(c) = 2 \cdot \mathsf{mask}_i(a) + 1, \\
		0 & \text{otherwise}.
	\end{cases}
\end{equation}

\begin{equation}
	\mathsf{add}_i(a, b, c) =
	\mathsf{layer}_i(a) \cdot \mathsf{layer}_{i-1}(b) \cdot \mathsf{layer}_{i-1}(c) \cdot \mathsf{lidx}_i(a, b, c).
\end{equation}

The purpose of decomposing $\mathsf{add}_i$ in this way is that the four terms of the product are each
dependent on disjoint subsets of the bits of $a, b,$ and $c$. So, to find the multilinear extension of
$\mathsf{add}_i$, it suffices to find the multilinear extensions of each of the four terms of the product.
Let's write the multilinear extensions explicitly, which will show the claim that the terms are dependent
on disjoint sets of bits and therefore that their product is multilinear:

\begin{equation}
	\widetilde{\mathsf{layer}}_i(a) = TODO.
\end{equation}

\begin{equation}
	\widetilde{\mathsf{lidx}_i}(a) = TODO.
\end{equation}

\end{document}

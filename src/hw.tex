\documentclass[11pt]{article}

\usepackage{amsfonts}
\usepackage{amsmath}
\usepackage[total={7in,9in}]{geometry}
\usepackage{amsthm}
\usepackage{graphicx}
\usepackage{lmodern}
\usepackage{hyperref}

\title{Berkeley ZK MOOC homework assignment}
\author{ \\ Morgan Thomas \\ Casper Association \\ morgan@casper.network }


\begin{document}

\maketitle

\section{Question 1}

\subsection{Part (a)}

\textbf{Question.} Give an expression for the multilinear extension
$\widetilde{\mathsf{eq}}_\ell : \{0,1\}^\ell \times \{0,1\}^\ell \to \{0,1\}$, that is defined
as follows:

\begin{equation}
  \mathsf{eq}_\ell(a, b) = \begin{cases}
	  1 & \text{if}\ a = b \\
	  0 & \text{otherwise}
  \end{cases}
\end{equation}

Argue that this polynomial can be evaluated at any point in time $O(\ell)$.

\textbf{Answer.} Let:

\begin{equation}
	\widetilde{\mathsf{eq}}_\ell(a,b) = \prod_{i=1}^\ell 1 - a_i + b_i.
\end{equation}

This polynomial agrees with $\mathsf{eq}$ on all
$(a,b) \in \{0,1\}^\ell \times \{0,1\}^\ell$. It is also multilinear,
as we can see by induction on $\ell$. When $\ell = 1$, it is clearly
multilinear. When $\ell > 1$, we can observe that

\begin{equation}
	\widetilde{\mathsf{eq}}_\ell(a,b) = (1 - a_\ell + b_\ell) \cdot \widetilde{\mathsf{eq}}_{\ell-1} = \widetilde{\mathsf{eq}}_{\ell-1}(a,b) - a_\ell \cdot \widetilde{\mathsf{eq}}_{\ell-1}(a,b) + b_\ell \cdot \widetilde{\mathsf{eq}}_{\ell-1}(a,b),
\end{equation}

and $\widetilde{\mathsf{eq}}_{\ell-1}$ does not refer to $a_\ell$ or $b_\ell$, so
therefore $\widetilde{\mathsf{eq}}_\ell$ is multilinear if $\widetilde{\mathsf{eq}}_{\ell-1}$ is.

These polynomials can be evaluated using $\ell$ additive inverse operations (one on each $a_i$), $2\ell$ addition operations, and $\ell-1$ multiplication operations, so, they can be evaluated using $O(\ell)$ operations. Said operations are $O(1)$ time, so overall these polynomials can be evaluated in $O(\ell)$ time.

\subsection{Part (b)}

\textbf{Question.} Assume that $n$ is a power of two. Give an expression for $\widetilde{\mathsf{mult}}$ that can be evaluated at any point in time $O(\log n)$. The multiplication layer consists of $n = 2^{d-1}$ multiplication gates, where the $j$-th multiplication gate at layer $d - 1$ has both in-neighbors equal to the $j$-th input gate at layer $d$.

\textbf{Answer.} For all positive integers $\ell$ and bit vectors $a,b,c$ of length $\ell$, let:

\begin{equation}
	\mathsf{eq}^3_\ell(a, b, c) =
	\begin{cases}
		1 & \text{if}\ a = b = c, \\
		0 & \text{otherwise}.
	\end{cases}
\end{equation}

Define the multilinear extension $\widetilde{\mathsf{eq}}^3_\ell$ of $\mathsf{eq}^3_\ell$ as follows:

\begin{equation}
	\widetilde{\mathsf{eq}}^3_\ell(a, b, c) =
	\prod_{i=1}^\ell (a_i \cdot b_i \cdot c_i) + ((1 - a_i) \cdot (1 - b_i) \cdot (1 - c_i)).
\end{equation}

It should be easy to check that this is an extension of $\mathsf{eq}^3_\ell$.
It is also multilinear because each term of the big product is multilinear and
no two terms refer to the same variable (and this reasoning can be further done out
as a proof by induction as in the previous answer). Also observe that this expression
can be evaluated in $O(\ell)$ time.

Let $n$ be the number of multiplication gates. Let $d$ be the number of layers
of the circuit. Consider that the number of nodes in the circuit is $2^d-1+n$,
where $2^d-1$ is the number of nodes in a perfect binary tree of height $d-1$,
and $n$ is the number of input nodes. Consdering that $n = 2^{d-1}$, it requires
$d+1$ bits to represent an index into the list of nodes. Let the bits in the
index bit strings be indexed in LSB first notation. Let $k$ be the number of bits
in $n$. Note that $k = O(\log n)$ and $d+1-n = O(d) = O(\log n)$.

Let:

\begin{equation}
	\widetilde{\mathsf{mult}}(a, b, c) =
	\widetilde{\mathsf{eq}}^3_k([a_1, ..., a_n], [b_1, ..., b_n], [c_1, ..., c_n])
	\cdot \widetilde{\mathsf{eq}}^3_{d+1-n}([b_{n+1}, ..., b_{d+1}], [c_{n+1}, ..., c_{d+1}], [0, ..., 0])
	\cdot \widetilde{\mathsf{eq}}_{d+1-n}([a_{n+1}, ..., a_{d+1}], [1, 0, ..., 0]).
\end{equation}

Observe that this defines the multilinear extension of $\mathsf{mult}(a, b, c)$,
and it can be evaluated in $O(\log n)$ since each of the subscripts on the
equality predicates is $O(\log n)$.

\end{document}
